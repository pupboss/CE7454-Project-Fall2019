\documentclass[conference]{IEEEtran}
\IEEEoverridecommandlockouts
% The preceding line is only needed to identify funding in the first footnote. If that is unneeded, please comment it out.
\usepackage{cite}
\usepackage{amsmath,amssymb,amsfonts}
\usepackage{algorithmic}
\usepackage{graphicx}
\usepackage{listings}
\usepackage{textcomp}
\usepackage{xcolor}
\def\BibTeX{{\rm B\kern-.05em{\sc i\kern-.025em b}\kern-.08em
    T\kern-.1667em\lower.7ex\hbox{E}\kern-.125emX}}
\begin{document}

\title{Project Plan of Object Detection in xxx}

\author{\IEEEauthorblockN{Jie Li, Meng Shen, Ruihang Wang\footnotemark*}
  \IEEEauthorblockA{\textit{School of Computer Science and Engineering} \\
    \textit{Nanyang Technological University}\\
    Email: \{jie006, meng005, ruihang001\}@e.ntu.edu.sg}
}
\maketitle

\renewcommand{\thefootnote}{\fnsymbol{footnote}}
\footnotetext[1]{Authors are sorted by initials.}

\section{Project Motivation}

Recently object detection is widely used in various fields, such as autonomous driving, video analytics and object tracking in filming. Thanks to the development of those object recognition techniques, there are many more possibilities in these industries. For example, auto driving could free drivers from fatigue of long distance driving, video analytics helps users detect potential targets with a pretty high accuracy and machines could run 7x24, object tracking assists professional photographers on eyes focus instead of focusing manually, moreover, it also introduces advanced filming skills to ordinary consumers for them to make fancy vlogs easily.

In summary, object detection will and always will be integrated into different industries, and will free users from repetitive works.

\section{Project Introduction}

So, in this project, we are going to build a neural network to detect {xxx} from {xxx} scenes, which currently is widely used in {xxx}, additionally, we plan to propose a new approach to {xxx} to get a better accuracy.

\section{Existing Solution}

So far many researchers have been putting efforts on this domain, before starting this project we also found several approaches, and some of them have got an awesome accuracy, they are YOLO, fast RCNN respectively.

In YOLO's implementation, YOLO employs a {xxxx} to {xxxx}, the advantages of {xxx} is {xxx}, but there are also some drawbacks, {xxx}.

As for the second method, fast RCNN {xxx}, {xxx}

\section{Proposed Solution}

According to the above information, we propose to {xxx}, we are trying to solve {xxx} problem properly, the estimated result/effection should be {xxx}.

\section{Project Milestones}

Project Milestones

\subsection{Data Acquisition \& Data Cleaning}

Data Acquisition Data Cleaning

\subsection{Apply Deep Learning Methods}

Apply Deep Learning Methods

\subsection{Results Analytics}

Results Analytics

\section{Conclusion}

First, we are going to crawl some data from {xxx} to support the trainning process, this should be done before {xxx}, then we will make some improvements based on the existing neural networks, finally an analytics report and the jupyter notebook will be delivered.

\begin{thebibliography}{00}
  
  \bibitem{xilinxdoc} S. Kumar, W. L. Hamilton, J. Leskovec, D. Jurafsky , ``Community Interaction and Conflict on the Web'' 2018.
  
\end{thebibliography}

\end{document}
