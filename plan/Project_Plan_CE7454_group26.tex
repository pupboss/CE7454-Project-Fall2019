\documentclass[conference]{IEEEtran}
\IEEEoverridecommandlockouts
% The preceding line is only needed to identify funding in the first footnote. If that is unneeded, please comment it out.
\usepackage{cite}
\usepackage{amsmath,amssymb,amsfonts}
\usepackage{algorithmic}
\usepackage{graphicx}
\usepackage{listings}
\usepackage{textcomp}
\usepackage{xcolor}
\def\BibTeX{{\rm B\kern-.05em{\sc i\kern-.025em b}\kern-.08em
    T\kern-.1667em\lower.7ex\hbox{E}\kern-.125emX}}
\begin{document}

\title{CE7454 Project: Simultaneous Prediction of Box-office and Movie Rating using Multimodal Neural Network}

\author{\IEEEauthorblockN{Jie Li, Meng Shen, Ruihang Wang\footnotemark*}
  \IEEEauthorblockA{\textit{School of Computer Science and Engineering} \\
    \textit{Nanyang Technological University}\\
    Email: \{jie006, meng005, ruihang001\}@e.ntu.edu.sg}
}
\maketitle

\renewcommand{\thefootnote}{\fnsymbol{footnote}}
\footnotetext[1]{Authors are sorted by initials.}

\section{Motivation}

The world's movie industy has witnessed an unprecedented boom in recent years.
With heavy investments on today's movie industry, reliable predictions of a movie's box-office
revenue before its theatrical release is essential for producers to reduce financial risk. From the
perspective of viewers, however, movie ratings is an imporatant factor of the quality of a movie. In
this project, we are interested in predicting both box-office revenue and movie rating in an early stage,
aiming to provide useful information for producers, distributors and viewers. 

\section{Project Introduction}

Although movie success has been considered as an unpredictable problme due to its social complexity, we 
are trying to develop our own model to predict and interpret the relationship among movie-related variables. 
Specifically, we will crawl raw data from IMDb. After preprocessing,
the original dataset will be used to train a multimodal neural network for simultaneous movie rating and box-office
revenue prediction. Finally, the potential relationship among movie-related variables will be analyzed and discussed for further investigation.

\section{Existing Solutions}

The idea of developing computational models to predict the financial success of a movie is not completely new.
Several studies have attampted to develop approaches for box-office revenue prediction. For example, movie trailers
were used as input data for a linear support vector machine (SVM) classifier to predict the opening-week box-office revenues [1].
In [2], several nonlinear regression algorithm were employed for building box-office forecasting models. Somdutta Basu built a
vanilla neural network for movie rating prediction [3].

The majority of existing solutions, however, focus on either box-office revenue or movie rating prediction alone. 
Without simultaneous providing useful information, these models fail to cater the demand of both movie producers and viewers.
Moreover, the performance of traditional prediction methods are typically limited in terms of the ability to process raw data and representation learning.

\section{Proposed Solution}

The project will present an approach for simultaneous prediction of box-office revenue and movie rating using 
special designed multimodal neural network. The raw data will be crawled from IMDb, including three type of inputs:

\begin{enumerate}
  \item Movie poster: image data containing potential information of a movie
  \item Movie genre: a binary vevtor of length 22
  \item Movie metadata: numerical data such as duration, budget, actors etc.
\end{enumerate}

To solve this task, we will clean and normalize these different data at first. Then we need to 
select effective neural networks for different kind of data and compare differnet optimization methods.
Finally, we will compare our model with different baseline architectures

\section{Milestones}

Project Milestones

\subsection{Data Acquisition \& Data Cleaning}

Data Acquisition Data Cleaning

\subsection{Apply Deep Learning Methods}

Apply Deep Learning Methods

\subsection{Results Analytics}

Results Analytics

\section{Conclusion}

First, we are going to crawl some data from {xxx} to support the trainning process, this should be done before {xxx}, then we will make some improvements based on the existing neural networks, finally an analytics report and the jupyter notebook will be delivered.

\begin{thebibliography}{00}
  
  \bibitem{xilinxdoc} S. Kumar, W. L. Hamilton, J. Leskovec, D. Jurafsky , ``Community Interaction and Conflict on the Web'' 2018.
  
\end{thebibliography}

\end{document}
